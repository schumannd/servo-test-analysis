\documentclass{scrartcl}
\usepackage{amsmath, amssymb, amsthm}
\usepackage{ngerman}
\usepackage[utf8]{inputenc}
\usepackage[T1]{fontenc}
\usepackage{geometry}
\usepackage{color}
\usepackage{graphicx}
\usepackage{hyperref}
\usepackage{listings}
\geometry{top=24mm,textheight=245mm,textwidth=160mm,heightrounded,right=27mm,head=14.5pt}


\newcommand{\todo}[1] {{\color{red}(TODO: #1)}}


%\renewcommand{\labelenumi}{\alph{enumi})}
\setlength{\parindent}{0mm}


\begin{document}
\pagestyle{plain}


\noindent
\begin{minipage}{0.66\textwidth}
Testen, Verifizieren, Analysieren\\
von Software WS 15/16\\
~\\
\textbf{Johannes Linke, David Schumann}
\end{minipage}
~
\begin{minipage}{0.30\textwidth}
Hasso-Plattner-Institut\\
Potsdam\\
\today
\end{minipage}


\begin{center}
 \huge \bf Servo
\end{center}

\section{About the Project (Application Survey)}

The Servo project is developed and maintained by the free-software community Mozilla in cooperation with Samsung. Servo is an experimental web browser layout engine that tries to create a highly parallel environment where many components (like HTML parsing, image decoding, rendering, etc.) are handled by isolated tasks. At the same time it is one of the biggest projects being developed in the Rust programming language, which is developed as well by Mozilla Research.

\subsection*{Main programming language: Rust}
While some dependencies and set up code is written in a different language, the main programming language remains Rust. This works well with the main focus of Servo: a highly parallel, safe and fast alternative to current engines. Rust was designed to be just that, by requiring the programmer to be very explicit with typing and error handling. This helps catching most errors at compilation time using static analysis. More details about the features of the Rust language are explained in section \ref{rust_features}.

\subsection*{Requirements, Specifications and Documentation under Mozilla}
Servo is an open source project hosted on GitHub\footnote{https://github.com/servo/servo}. As with most open source projects, the core developers have established a well documented GitHub workflow\footnote{https://github.com/servo/servo/wiki/Github-workflow}. It consists of the standard steps that go with any Github project where you don't have write access:
\begin{enumerate}
    \item Fork the original repository
    \item Write and commit your contribution into a branch in your fork.
    \item Open a Pull Request from the branch in your repository to the original one
    \item Discuss and fix any questions and suggestions by the reviewer.
    \item The reviewer states his approval, triggers build bot to run automated tests and merges the Pull Request
\end{enumerate}

Every Pull Request needs to pass the contribution checklist\footnote{\label{contributing}https://github.com/servo/servo/blob/master/CONTRIBUTING.md}. It covers technicalities such as: Every change needs to be accompanied by ``tests relevant to the fixed bug or new feature''\footnote{See last point at footnote \ref{contributing}.} which is to be evaluated by one of the reviewers from the core team. But it also includes the Rust Code of Conduct\footnote{http://www.rust-lang.org/conduct.html} to ensure a friendly and welcoming environment for all contributors.  \\

Further documentation can be found in the servo wiki\footnote{https://github.com/servo/servo/wiki} hosted on Github. In addition to the above points it includes protocol notes on the biweekly meetings, long and short term goals, and a vast amount of documents aimed at helping with development and testing.

\subsection*{Current testing status}

As of Jan 2015 the code base of Servo core (without dependencies) contains 110k lines of code. These are tested with about 10k tests that are devided into three different categories:
\begin{itemize}
    \item unit-tests:
    \item ref-tests:
    \item wpt-tests:
\end{itemize}

\todo{ continue here}

- GitHub's issue tracker\footnote{https://github.com/features} is used to track bugs and link the appropriate Pull Requests fixing them.
- Mozilla's mach system is used for testing.(Testing guide\footnote{https://github.com/servo/servo/wiki/Testing})


- Servo is scheduled in Q4 2015 for alpha release ``that just works'' (mobile, but also other platforms)\footnote{https://github.com/servo/servo/wiki/Roadmap}

\subsection*{What is your current personal involvement in the
application? Developer, tester, user, etc.?}
Aspiring user.

\subsection*{Document these findings in a summary for your project report at the end of the semester}
Done\\

\section{Initial Test Plan}
\subsection{Discuss \& evaluate five V\&V questions}
\begin{enumerate}
  \item When do verification and validation start? When are they complete?
  
  V \& V starts with the begining of the project. As with many open source project that are developed by the community, the stakeholder or ``customer'' is the community itself. Which means that developer and customer are often the same person. In the case of servo the software is validated in bi-weekly meetings\footnote{https://github.com/servo/servo/wiki/Meetings} where both short-term next steps and long term milestones are discussed.
  
  Verification is done through an extensive testing system. It is maintained by every developer as every code change needs to be accompanied by a test before it is merged into the code base. This test suite runs on every reviewed pull request, via two build servers (bors-servo\footnote{https://github.com/bors-servo} and buildbot\footnote{http://servo-buildbot.pub.build.mozilla.org/}). This ensures that no new bugs are introduced into the system.
  
V \& V will never be complete. Even after the release there will always be bigfixes or new features that need to be tested as well. One could argue that V \& V ends when the project gets abandoned but you can hardly call that a complete state.
  \item What particular techniques should be applied during development?
  
  The Rust language itself comes with a testing blahblah.
  \item How can we assess the readiness of a product?
  \item How can we control the quality of successive releases?
  \item How can the development process itself be improved?
\end{enumerate}
\subsection{Consider different test classifications, discuss \& evaluate these different test approaches w.r.t. your application:}

• Validation vs. Defect Testing

• Development, Release, User Testing

• Unit/Component,Integration,SystemTesting

\subsection{Consider available artefacts, discuss \& evaluate potentials w.r.t. coverage-based testing for your application}

\subsection{Develop \& document initial test plan based on your findings}





\newpage



\section{Test automation}


\subsection{cargo test}

Rust's package manager cargo already has tools for running tests. Testing is as easy as annotating any method with \texttt{\#[test]}, and then running \texttt{cargo test}. Cargo will then build a testing binary, run all annotated methods and print out a report that looks like this:


\begin{verbatim}
    Compiling gfx_tests v0.0.1 (file:///home/johannes/servo/components/servo)
    Running /home/johannes/servo/target/debug/deps/gfx_tests-9666be7e60be2090

running 6 tests
test text_util::test_transform_compress_none ... ok
test text_util::test_transform_compress_whitespace ... ok
test text_util::test_transform_compress_whitespace_newline_no_incoming ... ok
test text_util::test_transform_compress_whitespace_newline ... ok
test text_util::test_transform_discard_newline ... ok
test font_cache_thread::test_local_web_font ... ok

test result: ok. 6 passed; 0 failed; 0 ignored; 0 measured

\end{verbatim}

Any crashes inside the tests (e.g. by failing assertions) are interpreted as test failures, methods that exit normally are successes.

Cargo's builtin testing features are not meant for large-scale applications and test suites. There are some smaller libraries like stainless \footnote{\url{https://github.com/reem/stainless}} that are extending on cargo's features for example by adding setup and teardown methods, but such libraries haven't seen much use yet. 

Servo uses \texttt{cargo test} for running the unit tests. For running the more complex test suites, it builds the binaries with cargo but uses python scripts to execute the individual tests.

\todo{expand on mach how it works, the different test suites}


\subsection{Coverage tools}
\todo{}

\subsection{afl.rs?}
\todo{}





\section{ASA}



\subsection{ASA features of the Rust language} \label{rust_features}

Rust has an extensive set of static analysis techniques built-in. They became necessary to achieve the first design goal of the language, safety, while not impeding the second and third, speed and concurrency.

What follows is a list of classes of errors that are prevented by the Rust compiler. Most of them are allowed in other languages and become sources of programming errors. Finding such errors through static analysis is not always possible in other languages since they are not expressive enough and the programmer can't provide enough information in the code for an ASA tool to perform such an analysis.

\paragraph{Ownership.} In Rust, variables and objects are always "`owned"' by a scope (usually, a block of code delimited by curly braces) or another object. This ownership can be moved, for example by calling a method and handing over the object as a parameter, but the language provides no way to copy the ownership to an object. This way, the owner of an object can be statically determined, and, more importantly, each object can be deleted if its owning object is deleted or the owning scope ends. This way, it can be determined at compile-time when each object has to be deleted, completely eliminating use-after-free bugs, a common source for crashes and security vulnerabilities, and preventing most classes of memory leaks.

\paragraph{Borrowing.} To allow for more flexible movement of data, objects can be temporarily borrowed into other scopes. The compiler statically ensures that there exists either exactly one mutable borrow, through which the object can be mutated, or any number of immutable borrows. By preventing shared mutable state, a lot of programming errors are made impossible, for example data races. A data race occurs if two threads access the same data in an unsynchronized fashion and at least one of them writes. Since there can be only one thread writing at one specific data point at the same time, data races cannot happen. Iterator invalidation is another problem detected at compile time: Since iterating over a container mutably borrows the container, modifying the container, which would need to happen through a second mutable borrow, is not possible.

\paragraph{Safe memory accesses.} Likewise it is not possible to access uninitialized memory and, since pointers are not part of the language, null pointers do not exist as well.

\paragraph{Code-generating macros.} Rust features a powerful macro syntax. Unlike in C and C++, where macros simply perform text replacements, macros in Rust are hygienic, can generate different code depending on the type of input and take a variable number of parameters among other advantages. In fact, a library has been written that expands regular expressions to native Rust code, enabling a lot of optimization potential as well as compile-time syntax checking on the expression \footnote{\url{https://github.com/rust-lang-nursery/regex}}. Another library enables specifying parsers through a macro-based API \footnote{\url{https://github.com/Geal/nom}}. The resulting parsers are extremely fast and guaranteed to be safe.

\paragraph{Doctests.} As many other languages, Rust defines a syntax to document source code, and as in many other languages, the documentation can contain example code. It might look like this:

\small {
    \begin{verbatim}
    /// Constructs a new `Rc<T>`.
    /// # Examples
    /// ```
    /// use std::rc::Rc;
    /// let five = Rc::new(5);
    /// ```
    pub fn new(value: T) -> Rc<T> {
        // implementation goes here
    }
    \end{verbatim}
}

In Rust, however, these usage examples are executed and tested as part of \texttt{cargo test}. To simplify this process, the example code is automatically completed e.g. with a main function to make it a valid Rust program. More complex examples can be built, e.g. spanning multiple functions, and individual lines can be hidden to focus the user-visible documentation on the relevant parts. The full set of testing facilities, e.g. assertions, is available, making it possible to avoid outdated code examples altogether.

\paragraph{Other features.} There are several smaller features, restrictions and well-chosen defaults of the Rust language that improve the robustness of programs: For example, variables are immutable by default, encouraging a less error-prone programming style. Matches (the Rust-equivalent of switch-case statements) must be exhaustive, that is each possible value of the variable that is matched must be handled by a branch, making it impossible for the programmer to forget a case. And finally, runtime errors such as IO-errors must be explicitly handled by the programmer or will lead to an immediate crash of the program, as opposed to undefined behavior in e.g. C and C++.





\subsection{Additional tools}


\subsubsection{Lints reported by the Rust compiler}

Besides the language features outlined above, the rust compiler has some smaller ASA capabilities in the form of lints which are reported as compile warnings. They include unused imports and variables, unnecessarily mutable variables, dead code, variable and function names not following the naming conventions, and unconditional recursions. There are some more lints that are disabled by default like missing documentation of public interfaces and the usage of the \texttt{unsafe} keyword.

\subsubsection{rust-clippy}

Clippy \footnote{\url{https://github.com/Manishearth/rust-clippy}} is a compiler plugin that expands on the lints shipped with the Rust compiler and checks for common patterns that indicate inefficient, needlessly complex or unidiomatic Rust code. It is exceptionally easy to use since the complete procedure of setting it up consists of adding one line to the cargo configuration and main code file. Having done that, clippy will be run whenever any code in the project is compiled.\\
\\

Example of clippy output:
{
\scriptsize
\begin{verbatim}

servo/components/script/cors.rs:95:9: 98:10 warning: you seem to be trying to use match for destructuring
    a single pattern. Consider using `if let`, #[warn(single_match)] on by default
    
servo/components/script/cors.rs:95         match referer.scheme_data {
servo/components/script/cors.rs:96             SchemeData::Relative(ref mut data) => data.path = vec![],
servo/components/script/cors.rs:97             _ => {}
servo/components/script/cors.rs:98         };

/servo/components/script/cors.rs:95:9: 98:10 help: try
if let SchemeData::Relative(ref mut data) = referer.scheme_data { data.path = vec![] }

for further information visit https://github.com/Manishearth/rust-clippy/wiki#single_match

\end{verbatim}
}

The lints reported by clippy vary in type. Most of them are related to programming style or detecting patterns that were common in old Rust code but can be written simpler in newer versions of Rust. The \texttt{single\_match} lint in the example above is such a lint; the \texttt{if let} construct wasn't available until late 2014. Other lints detect type casts that may lead to loss of precision or truncation, boolean expressions that are tautologies, or obvious bugs like out of bounds accesses with constants. Clippy also includes two lints that warn on complex (as in deeply nested) types and methods with a high cyclomatic complexity.\\
\\
Servo already has clippy integrated, it can be run through \texttt{./mach clippy}. \todo{problems setting up clippy?} Running it, we made the following findings: 
\begin{itemize}
	\item Clippy reported 417 warnings.
    \item Of these, 122 were in autogenerated code.
    \item Another 11 were false positives. Several more might be false positives as well which we haven't investigated further.
    \item The vast majority of lints pointed at code that could be simplified.
    \item Some of these simplifications also led to potentially faster code.
    \item There were eight warnings on complex methods and two warnings on complex types.
    \item Not a single reported lint indicated a possible bug.
\end{itemize}

While working through the warnings, we fixed about 130 of them and made a pull request to servo \footnote{\url{https://github.com/servo/servo/pull/9123}} and reported several false positives on clippy \footnote{\url{https://github.com/Manishearth/rust-clippy/issues/528}} \footnote{\url{https://github.com/Manishearth/rust-clippy/issues/529}} \footnote{\url{https://github.com/Manishearth/rust-clippy/issues/532}}.


\todo{some conclusion on clippy}

\todo{some conclusion on ASA}

\todo{How good is ASA in detecting faults in your project?}


\subsection{Servo's custom lints}

Additional lints are relatively easy to add to the compilation process. While clippy makes use of this feature for generally applicable lints, servo itself adds a few lints that are specialized to servo's use case. For example, some objects in the servo codebase rely on being placed on the stack instead of the heap to provide certain safety guarantees. A compile-time lint prevents encapsulating these objects in containers that would place them on the heap. Furthermore, objects that have representations in JavaScript-controlled memory are statically checked for being either a root registered with the JavaScript garbage collector or contained within such a root.

\subsubsection{rust-fmt}
\todo{rust-fmt?}


\end{document}

