\documentclass{scrartcl}
\usepackage{amsmath, amssymb, amsthm}
\usepackage{ngerman}
\usepackage[utf8]{inputenc}
\usepackage[T1]{fontenc}
\usepackage{geometry}
\usepackage{color}
\usepackage{graphicx}
\usepackage{hyperref}
\usepackage{listings}
\geometry{top=24mm,textheight=245mm,textwidth=160mm,heightrounded,right=27mm,head=14.5pt}


%\renewcommand{\labelenumi}{\alph{enumi})}
\setlength{\parindent}{0mm}


\begin{document}
\pagestyle{plain}


\noindent
\begin{minipage}{0.66\textwidth}
Testen, Verifizieren, Analysieren\\
von Software WS 15/16\\
~\\
\textbf{Johannes Linke, David Schumann}
\end{minipage}
~
\begin{minipage}{0.30\textwidth}
Hasso-Plattner-Institut\\
Potsdam\\
\today
\end{minipage}


\begin{center}
 \huge \bf Servo
\end{center}

\section{Application Survey}


\subsection*{Which development paradigm? Which development
language(s)?}
Rust. Other packages with different langauges.
\subsection*{Requirements? Specification? Documentation? Other artefacts available?}
Servo is an open source project hosted on GitHub. As most of these project do, the core developers have established a certain GitHub workflow \footnote{https://github.com/servo/servo/wiki/Github-workflow}. Bla Bla. Every Pull Request needs to pass the contribution checklist \footnote{https://github.com/servo/servo/blob/master/CONTRIBUTING.md}\\

Also hosted on github the servo wiki \footnote{https://github.com/servo/servo/wiki} can be found. It documents blablabla

\subsection*{Current testing status, approach? Bug repositories?}
- GitHub's issue tracker \footnote{https://github.com/features} is used to track bugs and link the appropriate PRs fixing them to them.
- Mozilla's mach system is used for testing.(Testing guide\footnote{https://github.com/servo/servo/wiki/Testing})


- Servo is scheduled in Q4 2015 for alpha release ``that just works'' (mobile, but also other platforms)\footnote{https://github.com/servo/servo/wiki/Roadmap}
\subsection*{What is your current personal involvement in the
application? Developer, tester, user, etc.?}
Aspiring user.
\subsection*{Document these findings in a summary for your project report at the end of the semester}
Done\\

\section{Initial Test Plan}
\subsection{Discuss \& evaluate five V\&V questions}
\begin{enumerate}
  \item When do verification and validation start? When are they complete?
  \item What particular techniques should be applied during development?
  \item How can we assess the readiness of a product?
  \item How can we control the quality of successive releases?
  \item How can the development process itself be improved?
\end{enumerate}
\subsection{Consider different test classifications, discuss \& evaluate these different test approaches w.r.t. your application:}

• Validation vs. Defect Testing

• Development, Release, User Testing

• Unit/Component,Integration,SystemTesting

\subsection{Consider available artefacts, discuss \& evaluate potentials w.r.t. coverage-based testing for your application}

\subsection{Develop \& document initial test plan based on your findings}

\section{Test automation}

\subsection*{Find, experiment and evaluate test tools appropriate for your application}

\subsection*{Important parameters: development language, field of application}

\subsection*{Good overview of open source testing tools: http://www.opensourcetesting.org/}

\subsection*{Most important automation possibilities: Test case execution \& management, Coverage measurement}


\end{document}

